\documentclass[a4paper]{scrartcl}
\usepackage[l2tabu,orthodox]{nag}% Old habits die hard. All the same, there are commands, classes and packages which are outdated and superseded. nag provides routines to warn the user about the use of those.

\usepackage{listings, enumitem}
\usepackage{amsmath,tabu}
\usepackage[all,error]{onlyamsmath}% Error on deprecated math commands like $$ $$.
\usepackage{fixltx2e}
\usepackage[strict=true]{csquotes}

\usepackage{color}
\usepackage[colorlinks=true]{hyperref}
\usepackage{2111defs,2111theorems}
\title{COMP3121 Assignment1}
\author{Fiona Lin z5131048}

\newcommand{\ah}{\mathsf{a}}
\newcommand{\be}{\mathsf{b}}
\newcommand{\assn}[1]{{\color{red}\left\{#1\right\}}}
\newcommand{\length}[1]{\left|#1\right|}
\newcommand{\noof}[2]{\left\|#1\right\|_{#2}}
\newcommand{\VAR}{\pPkey{var}}
\def\L{\mathcal{L}}

\begin{document}
\maketitle

\paragraph{1. [20 marks]}
\label{sec:Question 1}
You're given an array $A$ of $n$ integers, and must answer a series of $n$ queries, each of the form: “How many elements $a$ of the array $A$ satisfy $L_k \leq a \leq R_k$ ?”, where $L_k$ and $R_k (1 \leq k \leq n)$ are some integers such that $L_k \leq R_k$ . Design an $O(n \log{n})$ algorithm that answers all of these queries.
\paragraph{Solution}
The $O(n\log{n})$ algorithm is as following:\\
Given an array $A$ of $n$ integers, and sort them in to ascending order. Therefore, perform binary search on Array $A$ for the indexes of $L_k$ and $R_k$; the binary search is a divide and conquor algorithm, it will achieve the $O(n\log{n})$ performance.\\
Since there is no operation required when $ n = 0$, there is always zero elements $a$ of the Array satisfy that condition.\\
Let's assuming $n > 0$;

\begin{itemize}
  \item {\bfseries Case a)} When $L_k$ and $R_k$ are elements in Array $A$, $L_k$ and $R_k$ are retrived by the binary search on Array $A$.
  The number of elements $a$ of Array $A$ satisfy $L_k \leq a \leq R_k$ is one plus the difference on the indexes of $L_k$ and $R_k$'.
  \item {\bfseries Case b)} When either of $L_k$ and $R_k$ are elements in Array $A$, $L_k$ and $R_k$ are never retrived by the binary search. Hence the $R_k$ should be terminate on the last index binary search on Array $A$, which is less than $R_k$. Similarly, the $L_k$ should be terminate on the last index binary search on Array $A$, which is greater than $L_k$
  The number of elements $a$ of Array $A$ satisfy $L_k \leq a \leq R_k$ is still one plus the difference on the indexes of $L_k$ and $R_k$.
\end{itemize}

\lstinputlisting[language=Python, firstline=1, lastline=26]{q1.py}
\paragraph{2. [20 marks, both (a) and (b) 10 marks each]}

\label{sec:Question 2}
You are given an array $S$ of $n$ integers and
another integer $x$.\\
(a) Describe an $O(n \log{n})$ algorithm (in the sense of the worst case performance) that determines whether or not there exist two elements in $S$ whose sum is exactly $x$.\\
(b) Describe an algorithm that accomplishes the same task, but runs in $O(n)$ expected (i.e., average) time.
Note that brute force does not work here, because it runs in $O(n^2)$ time.

\paragraph{Solution}
{\bfseries(a)}
\begin{enumerate}
  \item Take a element $k$ in the array and let $j$ be the difference of sum $x$ and element $k$
  \item Then binary sort Array $S$ is $O(n\log{n})$ and binary search $j$ in Array $S$. It is ideal to skip those $j$ outside the range of Array $S$. Performing this binary search is $O(n\log{n})$
  \item If $j$ is in Array $S$, then there exist the sum of two element exactly equal to $x$ in $S$, vice versa.
\end{enumerate}
\lstinputlisting[language=Python, firstline=1, lastline=30]{q2a.py}
{\bfseries(b)}
\begin{enumerate}
 \item Put every element $k$ of Array $S$ into Set $A$, this performs $O(n)$
 \item Take every element $k$ in the array and let $j$ be the difference of sum $x$ and element $k$. Since it is for every element $k$, this also takes $O(n)$
 \item Checking $j$ in Set $A$ takes $O(1)$. If $j$ is in Array $S$, then there exist the sum of two element exactly equal to $x$ in $S$, verse vice.
\end{enumerate}

\paragraph{The algorithm python}
\label{sec:code}
\lstinputlisting[language=Python, firstline=1, lastline=16]{q2b.py}


\paragraph{3. [20 marks, both (a) and (b) 10 marks each; if you solve (b) you do not have to
solve (a)]}
\label{sec:Question 3}
You are at a party attended by $n$ people (not including yourself), and you suspect that there might be a celebrity present. A celebrity is someone known by everyone, but does not know anyone except themselves. You may assume everyone knows themselves. Your task is to work out if there is a celebrity present, and if so, which of the n people present is a celebrity. To do so, you can ask a person $X$ if they know another person $Y$ (where you choose $X$ and $Y$ when asking the question).\\
\\*
(a) Show that your task can always be accomplished by asking no more than $3n-3$ such questions, even in the worst case.\\
\\*
(b) Show that your task can always be accomplished by asking no more than $3n-\floor{\log_2{n}}-2$ such questions, even in the worst case.

\paragraph{4. [20 marks, each pair 4 marks]}
\label{sec:Question 4}
Read the review material from the class website on asymptotic notation and basic properties of logarithms, pages 38-44 and then determine if $f(n) = \Omega(g(n))$, $f(n) = O(g(n))$ or $f (n) = \Theta(g(n))$ for the following pairs. Justify your answers. You might find the following inequality useful: \\
if $f (n), g(n), c > 0$ then $f (n) < c g(n)$; if and only if log$f(n) <$ log$c$ + log$g(n)$.
\begin{align*}
\begin{tabular}{|c | c |}
  \hline
  $f(n)$ & $g(n)$ \\
  \hline
  $\left(\log_2n\right)^2$ & $log_2\left(n^{log_2{n}}\right) + 2 log_2n$ \\
  \hline
  $n^{100}$ & $2^{n/100}$ \\
  \hline
  $\sqrt{n}$ & $2^{\sqrt{log_2 n}}$\\
    \hline
  $n^{1.001}$ & $n log_2 n$\\
  \hline
  $n^{(1+\sin{(\pi n/2))}/2n}$ &  $\sqrt{n}$\\
  \hline
\end{tabular}
\end{align*}
\paragraph{Solution}
\begin{enumerate}[label=(\alph*)]
  \item $f(n)=\left(\log_2n\right)^2,\ g(n)=\log_2\left(n^{\log_2{n}}\right) + 2*\log_2{n}$
  \begin{align*}
  &\ Since\ g(n)= \log_2\left(n^{\log_2{n}}\right) + 2*\log_2{n} = \log_2{n}*\log_2{n} + 2*\log_2{n} = \left(\log_2{n}\right)^2 + 2*\log_2{n}\\ &\
  When\ n_0\ =\ 1, f(1) = (\log_2{1})^2 = 1,\ g(1) = \left(\log_2{1}\right)^2 + 2*\log_2{1} = 0\\ &\
  let\ c > \frac{1}{3}\ and\ n = 2,\ f(2)-c*g(2) \leq 0 \Rightarrow c \geq \frac{1}{3}\\ &\
  \end{align*}
  \begin{align*}
  &\ Therefore\ c\geq \frac{1}{3}, \ n \geq n_0 = 1,\ f(n)=\left(\log_2{n}\right)^2 < c*g(n) = c*(\left(\log_2{n}\right)^2 + 2\log_2{n})\\ &\
  Hence,\ c\geq \frac{1}{3}, \ n \geq n_0 = 1,\ f(n) = O(g(n))
  \end{align*}
  \item $f(n)= n^{100},\ g(n)=2^{\frac{n}{100}}$
  \begin{align*}
  % solution f(n)= n^{100},\ g(n)=2^{\frac{n}{100}}
  &\ When\ c\ >\ 0\ and\ n_0\ = \ 1\ ,\\ &\
  \quad f(n_0) - c*g(n_0)\ =\ n_0^{100}-c*(2^\frac{n_0}{100})  = 1^{100}- c*(2^{\frac{1}{100}}) \geq 0\ ,\\ &\
  \Rightarrow\ c\ =\ 2^{\frac{-1}{100}} \approx 0.9930.924... > 0\\ &\
  Therefore,\ when\ c\ >\ 0\ and\ n \geq \ n_0\ = \ 1,\ 0\ \leq\ c*g(n)\ \leq\ f(n)\\ &\
  Hence, \ when\ c\ >\ 0\ and\ n \geq \ n_0\ = \ 1,\ f(n) = \Omega(g(n))
  \end{align*}
  \item $f(n)=\sqrt{n},\ g(n)=2^{\sqrt{log_2 n}}$
  \begin{align*}
  % solution f(n)=\sqrt{n},\ g(n)=2^{\sqrt{\log_2 n}}
  &\ Since\ f(n)\ -\ g(n)=\sqrt{n}\ -\ 2^{\sqrt{\log_2 n}} \geq 0\ \Rightarrow n\ >\ 16\\ &\
  Therefore,\ exist\ c_1=c_2=1\ and\ n\ >\ n_0\ =\ 16,\ 0\ \leq\ c_1*g(n)\ \leq\ f(n)\ \leq\ c_2*g(n)\\ &\
  Hence,\ f(n)\ =\ \Theta(g(n))
  \end{align*}
  \item $f(n)=n^{1.001},\ g(n)=n*\log_2{n}$
  \begin{align*}
  % solution f(n)=n^{1.001},\ g(n)=n log_2 n
  &\ When\ n\ =\ 1,\\ &\ \quad g(1)-f(1)=n*\log_2{n}-n^{1.001}=\log_2{1}-1^{1.001}=0-1=-1<0 \\&\
  Then\ when\ n\ \geq\ 2, \\ &\ \quad
  c*g(n)-f(n)= c*(n*\log_2{n})-n^{1.001}>0\ \Rightarrow\ c > 2^{0.001} \approx 1.00069339...\\&\
  Therefore, when\ c=2\ and\ n \geq 2,\ c*g(n) \geq f(n)\\&\
  Hence,\ f(n) = O(g(n))
  \end{align*}
  \item $f(n)=n^{\frac{1+\sin(\frac{\pi n}{2})}{2n}},\ g(n)=\sqrt{n}$
  \begin{align*}
  % solution f(n)=n^{(1+sin(\frac{\pi n}{2}))/2n},\ g(n)=\sqrt{n}
  &\ If\ f(n),\ g(n),\ c\ >\ 0\ then\ f(n)\ <\ c*g(n);\ if\ and\ only\ if\ \log{f(n)}\ <\ \log{c}\ +\ \log{g(n)}\\&\
  When\ c\ =\ 1,\ \log{c} = 0,\\&\
  \log{f(n)}\ =\ \log{\sqrt{n}},\\&\ \log{g(n)}=\log{\sqrt(n)}^{\frac{1+\sin{\frac{n\pi}{2}}}{n}}\\&\
  Then\ we\ have\\&\
  \log{c}+\log{g(n)}-\log{f(n)}=\log{c}+\log{\sqrt(n)}^{\frac{1+\sin{\frac{n\pi}{2}}}{n}}-\log{\sqrt{n}}=\log{c}+(\frac{n+1+\sin{\frac{n\pi}{2}}}{n})\log{\sqrt(n)}\\&\
  We\ need\ \log{c}+\log{g(n)}-\log{f(n)}\ >\ 0,\ then\ we\ can\ solve\ n\ with:\\ &\
  \log{c}+\log{g(n)}-\log{f(n)}=(\frac{n+1+\sin{\frac{n\pi}{2}}}{n})\log{\sqrt(n)}>0\Rightarrow\ n\ \geq\ 1 \\&\
  Therefore,\ c=1,\ n\ >\ n_0=1,\ \log{c}+\log{g(n)}-\log{f(n)} \iff 0 \leq f(n)\ <\ c*g(n)\\&\
  Hence, \ c=1,\ n\ >\ n_0=1, f(n)=O(n)
  \end{align*}
\end{enumerate}
\paragraph{5. [20 marks, each recurrence 5 marks]}
\label{sec:Question 5}
Determine the asymptotic growth rate of the solutions to the following recurrences. If possible, you can use the Master Theorem, if not, find another way of solving it.
\begin{enumerate}[label=(\alph*)]
  \item $T (n) = 2T (n/2) + n(2 + \sin{n})$
  \item $T (n) = 2T (n/2) + \sqrt{n} + \log{n}$
  \item $T (n) = 8T (n/2) + n^{\log{n}}$
  \item $T (n) = T (n − 1) + n$
\end{enumerate}
\paragraph{Solution}
\begin{enumerate}[label=(\alph*)]
  \item $T (n) = 2T (n/2) + n(2 + \sin{n})$
\begin{align*}
  &\ As\ a = 2\ and\ b = 2,\quad then \quad n^{\log_b{a}} = n^{\log_2{2}} = n\\
  &\ And\ f(n) = n(2 + \sin{n}),\quad then\\ &\
  \qquad \sin{n} \in \left[-1, 1\right]\ \Rightarrow 2+\sin{n} \in \left[1,\ 3\right] \Rightarrow f(n) = n\left(2 + \sin{n}\right) \in \left[n,\ 3n\right]\\ &\
  Since\quad n^{\log_b{a}} = n \quad and \quad f(n) \in \left[n,\ 3n\right], f(n) = n(2 + \sin{n}) = \Theta(n)\\ &\
  Condition\ of\ case\ 2\ is\ satisfied;\ and\ so:\\
  &\ \qquad T(n)=\Theta\left({n^{\log_2{2}}\log{n}}\right) =\Theta\left({n\log{n}}\right)
\end{align*}
\item $T (n) = 2T (n/2) + \sqrt{n} + \log{n}$
\begin{align*}
  &\ As\ a = 2\ and\ b = 2,\quad then \quad n^{\log_b{a}} = n^{\log_2{2}} = n\\
  &\ When\ n \in \left(1, \infty\right),\ f(n) \in \left(0,\ \infty\right], f(n)\ is\ a\ non\text{-}decreasing\ function\\ &\
  Since\quad n^{\log_b{a}} = n \quad and \quad f(n) = \sqrt{n} + \log{n}\\ &\
  f(n) = \sqrt{n} + \log{n} \\ &\
  O? \Theta ? \Omega ? (n^{4+-\epsilon})
\end{align*}
\item $T (n) = 8T (n/2) + n^{\log{n}}$
\begin{align*}
  &\ As\ a = 8\ and\ b = 2,\quad then \quad n^{\log_b{a}} = n^{\log_2{8}} = n^4\\ &\
  Since\quad g(n)=n^{\log_b{a}}\ =\ n^4 \quad and \quad f(n) = n^{\log{n}}\\ &\
  =\ \Theta(n^{4-\epsilon})\ for\ any\ \epsilon < 3\\ &\
  O? \Theta ? \Omega ? (n^{4+-\epsilon})
\end{align*}
\item $T (n) = T (n - 1) + n$
\begin{align*}
  &\ To\ apply\ Master\ Theorem,\ a \geq 1\ and\ b > 1\ .\\ &\
  Since\ b = 1,\ it\ is\ not\ applicable\ to\ determine\ asymptotic\ growth\ rate\ of\\ &\
  this\ recurrences.
\end{align*}
\end{enumerate}
\end{document}
