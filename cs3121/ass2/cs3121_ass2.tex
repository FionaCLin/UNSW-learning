\documentclass[a4paper]{scrartcl}
\usepackage[l2tabu,orthodox]{nag}% Old habits die hard. All the same, there are commands, classes and packages which are outdated and superseded. nag provides routines to warn the user about the use of those.
\usepackage{listings, enumitem}
\usepackage{amsmath,tabu}
\usepackage[all,error]{onlyamsmath}% Error on deprecated math commands like $$ $$.
\usepackage{fixltx2e}
\usepackage[strict=true]{csquotes}

\usepackage{color}
\usepackage[colorlinks=true]{hyperref}
\usepackage{2111defs,2111theorems}

\title{COMP3121 Assignment1}
\author{Fiona Lin z5131048}

\usepackage{graphicx}
\usepackage{subcaption}

\newcommand{\ah}{\mathsf{a}}
\newcommand{\be}{\mathsf{b}}
\newcommand{\assn}[1]{{\color{red}\left\{#1\right\}}}
\newcommand{\length}[1]{\left|#1\right|}
\newcommand{\noof}[2]{\left\|#1\right\|_{#2}}
\newcommand{\CON}{\pPkey{convolution}}
\newcommand{\FFT}{\pPkey{FFT}}
\newcommand{\IFFT}{\pPkey{IFFT}}
\def\L{\mathcal{L}}
\begin{document}
\maketitle
\paragraph{1. [20 marks]}
\label{sec:Question 1}
You are given two polynomials,
\begin{align*} &\
  P_A(x) = A_0 + A_3x^3+A_6x^6\\ &\
\qquad\qquad\text{and}\\ &\
  P_B(x) = B_0 + B_3x^3+B_6x^6+B_9x^9
\end{align*} 
where all $A_i'$s and $B_j'$s are large numbers. Multiply these two polynomials using only 6 large number Multiplicaitons.
\paragraph{Solution}
Let $P_C(y)=P_A(y) \cdot P_B(y)$ and $y =x^3$, then we have\\ 
$P_A(y) = A_0 + A_3 y + A_6 y^2 \quad and\quad 
P_B(y) = B_0 + B_3 y + B_6 y^2 + B_9 y^3$\\
Since the product polynomial $P_C(y) = P_A(y)\cdot P_B(y)$ is of degree 5, we need 6 value to uniquely determine $P_C(y)$. Let $y = -2,-1,0,1,2,3$; we have
\begin{align*}
&\ 
P_A(-2) = A_0 + (-2) A_3 + (-2)^2  A_6  = A_0 - 2A_3 + 4A_6 \\ &\
P_B(-2) = B_0 + (-2) B_3 + (-2)^2  B_6 + (-2)^3  B_9 = B_0 -2B_3 + 4B_6 - 8B_9\\ &\
P_A(-1) = A_0 + (-1) A_3 + (-1)^2  A_6  = A_0 - 1A_3 +  A_6 \\ &\
P_B(-1) = B_0 + (-1) B_3 + (-1)^2  B_6 + (-1)^3  B_9 = B_0 -1B_3 +  B_6 -  B_9\\ &\
P_A(0) = A_0 + (0) A_3 + (0)^2 A_6 = A_0 \\ &\
P_B(0) = B_0 + (0) B_3 + (0)^2 B_6 + (0)^3 B_9= B_0\\ &\
P_A(1) = A_0 + (1) A_3 + (1)^2 A_6 = A_0 +  A_3 +   A_6 \\ &\
P_B(1) = B_0 + (1) B_3 + (1)^2 B_6 + (1)^3 B_9= B_0 +  B_3 +  B_6 +   B_9\\ &\
P_A(2) = A_0 + (2) A_3 + (2)^2 A_6 = A_0 + 2A_3 +  4A_6 \\ &\
P_B(2) = B_0 + (2) B_3 + (2)^2 B_6 + (2)^3 B_9= B_0 + 2B_3 + 4B_6 + 8 B_9\\ &\
P_A(3) = A_0 + (3) A_3 + (3)^2 A_6 = A_0 + 3A_3 +  9A_6 \\ &\
P_B(3) = B_0 + (3) B_3 + (3)^2 B_6 + (3)^3 B_9= B_0 + 3B_3 + 9B_6 + 27B_9\\ &\
\end{align*}
Thus, if we present the product $P_C(y)=P_A(y)P_B(y)$ in the coefficient form as\\
$P_C(y) = C_0 +C_1 y +C_2 y^2 +C_2 y^2 +C_3 y^3 +C_4 y^4 +C_5 y^5 
$\\
We get
\begin{align*}
C_0 - 2C_1 + 4 C_2 - 8  C_3 + 16 C_4 - 32  C_5 + 64  C_6 = P_C(-2) = P_A(-2)P_B(-2)\\
C_0 -  C_1 +   C_2 -    C_3 +    C_4 -     C_5 +     C_6 = P_C(-1) = P_A(-1)P_B(-1)\\
C_0 =  P_C(0)  = P_A(0)P_B(0)\\
C_0 +  C_1 +   C_2 +    C_3 + C_4    +     C_5 +     C_6 =  P_C(1) = P_A(1)P_B(1) \\
C_0 + 2C_1 + 4 C_2 + 8  C_3 + 16C_4  + 32  C_5 + 64  C_6 =  P_C(2) = P_A(2)P_B(2) \\
C_0 + 3C_1 + 9 C_2 + 27 C_3 + 81 C_4 + 243 C_5 + 729 C_6 =  P_C(3) = P_A(3)P_B(3) \\
\end{align*}
Solving this system of linear equations for $C_0, C_1, C_2, C_3, C_4, C_5$ we obtain
\begin{align*}
&\
C_0 =  P_C(0) \\ &\
C_1 = \frac{60P_C(3)-15P_C(2)+2P_C(1)-20P_C(0)-30P_C(-1)+3P_C(-2)}{60}\\ &\
C_2 = -\frac{-16P_C(3)+P_C(2)+30P_C(0)-16P_C(-1)+P_C(-2)}{24}\\ &\
C_3 = -\frac{14P_C(3)-7P_C(2)+P_C(1)-10P_C(0)+P_C(-1)+P_C(-2)}{24}\\ &\
C_4 = \frac{-4P_C(3)+P_C(2)+6P_C(0)-4P_C(-1)+P_C(-2)}{24}\\ &\
C_5 = -\frac{-10P_C(3)+5P_C(2)-P_C(1)+10P_C(0)-5P_C(-1)+P_C(-2)}{120}\\ &\
\end{align*}
Multiply these two polynomials using only these 6 large number multiplicaitons $C_0, C_1, C_2, C_3, C_4, C_5$.
\paragraph{2.}
\label{sec:Question 2}
\begin{enumerate}[label=(\alph*)]
  \item {\bfseries[5 marks]} Multiply two complex numbers $(a + ib)$ and $(c + id)$ (where $a, b, c, d$ are all real numbers) using only 3 real number multiplications.
  \item {\bfseries[5 marks]} Find $(a + ib)^2$ using only two multiplications of real numbers.
  \item {\bfseries[10 marks]} Find the product $(a + ib)^2(c + id)^2$ using only five real number multiplications.
\end{enumerate}
\paragraph{Solution}
\begin{enumerate}[label=(\alph*)]
  \item Let $Z_0=a + ib,\ Z_1=c + id$, and we have $(a + b)(c + d) = ac + bd + bc + ad$, then
  \begin{align*}
   (a + ib)(c + id)=ac - bd + i(ad+bc) =ac - bd + i((a+b)(c+d)-ac-bd)
  \end{align*}
  Therefore, multiply two complex numbers $(a + ib)$ and $(c + id)$ using only 3 real number multiplications $ac, bd, (a + b)(c + d)$.
  \item  As $z_0^2=(a + ib)^2=a^2 -b^2 + 2iab =(a+b)(a-b) + 2iab$
   Therefore, using only two multiplications of real numbers,$(a+b)(a-b)  ,ab$.
  \item
  From previously, we have $z_0z_1$ using 3 multiplicaitons and $z_0^2$ using 2 multiplicaitons.
  So we can use evaluate $z_0^2z_1^2 =(a + ib)^2(c + id)^2= (z_0z_1)^2$ with $z_0z_1$ using 3 multiplicaitons and then apply the result $z_0^2$ with 2 multiplicaitons.
  Thus, we can compute $(a + ib)^2(c + id)^2$ using only 5 multiplicaitons.
\end{enumerate}
\paragraph{3.}
\label{sec:Question 3}
\begin{enumerate}[label=(\alph*)]
\item {\bfseries[2 marks]} $Revision$: Describe how to multiply two $n$-degree polynomials together in $O(n\log{n})$ time, using the Fast Fourier Transform (FFT). You do not need to explain how FFT works – you may treat it as a black box.
\item In this part we will use the Fast Fourier Transform (FFT) algorithm described in class to multiply multiple polynomials together (not just two). Suppose you have $K$ polynomials $P1$, ... ,$PK$ so that
\begin{align*}
  degree(P1 ) + ...  + degree(PK ) = S
\end{align*}
\begin{enumerate}[label=(\roman*)]
  \item {\bfseries[6 marks]} Show that you can find the product of these $K$ polynomials in $O(KS\log{S})$ time.
  Hint: How many points do you need to uniquely determine an $S$-degree polynomial?
  \item {\bfseries[12 marks]} Show that you can find the product of these $K$ polynomials in $O(S\log{S}\log{K})$ time.
  Hint: consider using divide-and-conquer; a tree which you used in the previous assignment might be helpful here as well. Also, remember that if $x,y,z$ are all positive, then $\log{(x + y)} < \log{(x + y + z)}$
\end{enumerate}
\end{enumerate}
\paragraph{Solution}
\begin{enumerate}[label=(\alph*)]
\item Multiply two $n$-degree polynomials actually convert the coefficients of those polynomials into 2 vectors and perform convolution on these two verctors.\\ 
Naive convolution take $O(n^2)$ time complexity, however, 
convolution can be optimised using the Fast Fourier Transform (FFT) algorithm:\\
$ \CON(a,b) = \sqrt{n}(\IFFT(\FFT(a) \cdot \FFT(b)))$\\
each FFT takes $O(\log{n})$ time and the product of 2 FFT is elementwise product(also know as Hadamard product) and it's $O(n)$, hence the overall time complexity is $O(n\log{n})$ time.
\item In this part we will use the Fast Fourier Transform (FFT) algorithm described in class to multiply multiple polynomials together (not just two). Suppose you have $K$ polynomials $P1$, ... ,$PK$ so that
\begin{enumerate}[label=(\roman*)]
  \item {\bfseries[6 marks]} Show that you can find the product of these $K$ polynomials in $O(KS\log{S})$ time.
  Hint: How many points do you need to uniquely determine an $S$-degree polynomial?
  \item {\bfseries[12 marks]} Show that you can find the product of these $K$ polynomials in $O(S\log{S}\log{K})$ time.
  Hint: consider using divide-and-conquer; a tree which you used in the previous assignment might be helpful here as well. Also, remember that if $x,y,z$ are all positive, then $\log{(x + y)} < \log{(x + y + z)}$
\end{enumerate}
\end{enumerate}
\paragraph{4. [20 marks, each pair 4 marks]}
\label{sec:Question 4}
Let us define the Fibonacci numbers as $F_0=0,\ F_1=1$ and $F_n=F_{n-1}+F_{n-2}$ for all $n\geq2$. Thus, the Fibonacci sequence looks as follows: $0, 1, 1, 2, 3, 5, 8, 13, 21, . . .$
\begin{enumerate}[label=(\alph*)]
  \item{\bfseries[5 marks]} Show, by induction or otherwise, that
\begin{align*}
  \begin{pmatrix}
   F_{n+1} & F_n \\
   F_n & F_{n-1} \\
  \end{pmatrix}
  =
  \begin{pmatrix}
    1 & 1 \\
    1 & 0 \\
  \end{pmatrix}^{\!n}
\end{align*}
  for all integers $n\geq1$
  \item {\bfseries[15 marks]} Hence or otherwise, give an algorithm that finds Fn in $O(\log{n})$ time.
\end{enumerate}
\paragraph{Solution}



\paragraph{5.}
\label{sec:Question 5}
Your army consists of a line of $N$ giants, each with a certain height. You must designate precisely $L \leq N$ of them to be leaders. Leaders must be spaced out across the line; specifically, every pair of leaders must have at least $K \geq 0$ giants standing in between them. Given $N,L,K$ and the heights $H[1..N]$ of the giants in the order that they stand in the line as input, find the $maximum$ height of the $shortest$ leader among all valid choices of $L$ leaders. We call this the optimisation version of the problem.\\
\\*
For instance, suppose $N = 10,L = 3,K = 2$ and $H = [1,10,4,2,3,7,12,8,7,2].$ Then among the 10 giants, you must choose 3 leaders so that each pair of leaders has at least 2 giants standing in between them. The best choice of leaders has heights 10, 7 and 7, with the shortest leader having height 7. This is the best possible for this case.
\begin{enumerate}[label=(\alph*)]
  \item {\bfseries[8 marks]} In the $decision$ version of this problem, we are given an additional integer $T$ as input. Our task is to decide if there exists some valid choice of leaders satisfying the constraints whose shortest leader has height no less than $T$ .
  \item {\bfseries[12 marks]} Hence, show that you can solve the optimisation version of this problem in $O(N\log{N})$ time.
\end{enumerate}
\paragraph{Solution}
\end{document}
