\documentclass[a4paper]{scrartcl}
\usepackage[l2tabu,orthodox]{nag}% Old habits die hard. All the same, there are commands, classes and packages which are outdated and superseded. nag provides routines to warn the user about the use of those.
\usepackage{listings, enumitem}
\usepackage{amsmath,tabu}
\usepackage[all,error]{onlyamsmath}% Error on deprecated math commands like $$ $$.
\usepackage{fixltx2e}
\usepackage[strict=true]{csquotes}

\usepackage{color}
\usepackage[colorlinks=true]{hyperref}
\usepackage{2111defs,2111theorems}

\title{COMP3121 Assignment1}
\author{Fiona Lin z5131048}

\usepackage{graphicx}
\usepackage{subcaption}

\newcommand{\ah}{\mathsf{a}}
\newcommand{\be}{\mathsf{b}}
\newcommand{\assn}[1]{{\color{red}\left\{#1\right\}}}
\newcommand{\length}[1]{\left|#1\right|}
\newcommand{\noof}[2]{\left\|#1\right\|_{#2}}
\newcommand{\CON}{\pPkey{convolution}}
\newcommand{\FFT}{\pPkey{FFT}}
\newcommand{\IFFT}{\pPkey{IFFT}}
\def\L{\mathcal{L}}
\begin{document}
\maketitle
\paragraph{Mid-06S1} 1) 
\begin{enumerate}[label=\alph*]
  \item $T(n) = 16T(\frac{n}{4}) + n $
  \begin{align*}
    n^{\log_b{a}} = n^{\log_4{16}} = n^2
    f(n) = n = O(n^{2-\epsilon}) (case 1) \Rightarrow T(n)=\Theta(n^2)    
  \end{align*}
  \item $T(n) = T(\frac{3n}{8}) + n $
  \begin{align*}
    n^{\log_b{a}} = n^{\log_{\frac{8}{3}}{1}} = n^0
    f(n) = n = \Omega(n^{0+\epsilon}) (case 3) \Rightarrow T(n) = \Theta(n)
  \end{align*}
  \item $T(n) = \sqrt{n}T(\sqrt{n}) + n\log_2{n}$
  
  \item $T(n) = T(n-1) + n + \log_2{n} + 1$
\end{enumerate}
2. a) 
First, merge sort array A takes $O(n\log{n})$ even worst case
Then for all element n pair up another n' from A takes $O(n^2)$
and then compute z from $z^2 = x^2+y^2$ take O(1); then binary search 
z in A takes $O(\log{n})$
Hence worst case takes $O(n^2\log{n})$ 
b) First, merge sort array A takes $O(n\log{n})$ even worst case.
Then


3) Multiplies 2 cubic polynomials of degree 3 will product a polynomials of degree
6 which can be uniquely identified by 7 real numbers coefficients.
The algorithms will split the coefficients of those polynomials into half to perform 
FFT and then combine 2 halfs using IFFT recursively. 
Finally we will have the 7 real number multiplications for the product polynomials

4) let  $y=x^8$, so $f(x)=x^16 + 8x^8 +1\Rightarrow f(y)=y^2+8y+1$.
Therefore $f(y)=(y+4)^2-15=(x+4-\sqrt{15})(x+4+\sqrt{15})$ will have 2 distinct values. 
So $f(x)$ will have distinct values $16 = 8*2$ 


\end{document}
